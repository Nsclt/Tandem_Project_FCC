\section{Conclusion}

The cross section study in section \ref{subsec:cross} shows, that \texttt{MadGraph} 
does not provide a model yet, that displays the expected resonant behavior of the 
cross section as a function of the centre-of-mass energy. \texttt{MadGraph} is able 
to calculate at next-to-leading order. To archieve results comparable to the 
results from the Conceptual Design Report \cite{CDR}, a scaling to NNNLO needs 
to be performed. Due to the difference in the obtained and expected behavior of 
the cross section, further results in the coefficient studies need to be taken 
with caution. \par \medskip

The results from the coefficient studies display the incredible precision and 
accuracy that the Future Circular Collider is able to archieve. Using measurements and 
data recorded at the FCC, a sigificant amount of beyond-the-standard-model physics 
could be researched to higher accuracy that nowadays. More exclusions and 
more narrow ranges in the coefficient strength of 
$C_\text{tW}$ and $C_\text{tZ}$ can be made with multiple measurements included. 
Therefore it would be interesting to combine recent measurements with 
predictions from the Future Circular Collider. \par \medskip

The Future Circular Collider as a possible future accelerator provides many 
opportunities to test the standard model further. The coefficient study in this 
report only shows one of many possiblities that can be realized.