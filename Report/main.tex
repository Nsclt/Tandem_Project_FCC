\documentclass[
  bibliography=totoc,     % Literatur im Inhaltsverzeichnis
  captions=tableheading,  % Tabellenüberschriften
  titlepage=firstiscover, % Titelseite ist Deckblatt
]{scrartcl}

% Paket float verbessern
\usepackage{scrhack}

% Warnung, falls nochmal kompiliert werden muss
\usepackage[aux]{rerunfilecheck}


% unverzichtbare Mathe-Befehle
\usepackage{amsmath}
% viele Mathe-Symbole
\usepackage{amssymb}
% Erweiterungen für amsmath
\usepackage{mathtools}
%Fallunterscheidungen
\usepackage{cases}
% Fonteinstellungen
\usepackage{fontspec}
% Latin Modern Fonts werden automatisch geladen
% Alternativ zum Beispiel:
%\setromanfont{Libertinus Serif}
%\setsansfont{Libertinus Sans}
%\setmonofont{Libertinus Mono}

% Wenn man andere Schriftarten gesetzt hat,
% sollte man das Seiten-Layout neu berechnen lassen
\recalctypearea{}

% deutsche Spracheinstellungen
\usepackage[main=ngerman]{babel}


\usepackage[
  math-style=ISO,    % ┐
  bold-style=ISO,    % │
  sans-style=italic, % │ ISO-Standard folgen
  nabla=upright,     % │
  partial=upright,   % ┘
  warnings-off={           % ┐
    mathtools-colon,       % │ unnötige Warnungen ausschalten
    mathtools-overbracket, % │
  },                       % ┘
]{unicode-math}

% traditionelle Fonts für Mathematik
%\setmathfont{Latin Modern Math}
% Alternativ zum Beispiel:
%\setmathfont{Libertinus Math}

%\setmathfont{XITS Math}[range={scr, bfscr}]
%\setmathfont{XITS Math}[range={cal, bfcal}, StylisticSet=1]

% Zahlen und Einheiten
\usepackage[
  locale=DE,                   % deutsche Einstellungen
  separate-uncertainty=true,   % immer Fehler mit \pm
  per-mode=symbol-or-fraction,
  range-phrase = -, % / in inline math, fraction in display math
]{siunitx}

% chemische Formeln
\usepackage[
  version=4,
  math-greek=default, % ┐ mit unicode-math zusammenarbeiten
  text-greek=default, % ┘
]{mhchem}

% richtige Anführungszeichen
\usepackage[autostyle]{csquotes}

% schöne Brüche im Text
\usepackage{xfrac}

% Farben zum Markieren 
\usepackage{xcolor}

% Standardplatzierung für Floats einstellen
\usepackage{float}
\floatplacement{figure}{htbp}
\floatplacement{table}{htbp}

% Floats innerhalb einer Section halten
\usepackage[
  section, % Floats innerhalb der Section halten
  below,   % unterhalb der Section aber auf der selben Seite ist ok
]{placeins}

% Seite drehen für breite Tabellen: landscape Umgebung
\usepackage{pdflscape}

% Captions schöner machen.
\usepackage[
  labelfont=bf,        % Tabelle x: Abbildung y: ist jetzt fett
  font=small,          % Schrift etwas kleiner als Dokument
  width=0.9\textwidth, % maximale Breite einer Caption schmaler
]{caption}
% subfigure, subtable, subref
\usepackage{subcaption}

% Grafiken können eingebunden werden
\usepackage{graphicx}
% größere Variation von Dateinamen möglich
%\usepackage{grffile}
\usepackage{wrapfig}
% schöne Tabellen
\usepackage{booktabs}

% Verbesserungen am Schriftbild
\usepackage{microtype}

% Literaturverzeichnis
\usepackage[
  backend=biber,
]{biblatex}
% Quellendatenbank
\addbibresource{lit.bib}
\addbibresource{programme.bib}

% Hyperlinks im Dokument
\usepackage[
  german,
  unicode,        % Unicode in PDF-Attributen erlauben
  pdfusetitle,    % Titel, Autoren und Datum als PDF-Attribute
  pdfcreator={},  % ┐ PDF-Attribute säubern
  pdfproducer={}, % ┘
]{hyperref}
% erweiterte Bookmarks im PDF
\usepackage{bookmark}

% Trennung von Wörtern mit Strichen
\usepackage[shortcuts]{extdash}

% Eigene Befehle
\newcommand{\leadingzero}[1]{\ifnum #1<10 0\the#1\else\the#1\fi}
\newcommand{\todayshort}{\leadingzero{\day}.\leadingzero{\month}.\the\year}

\setlength{\parindent}{0pt}

\begin{document}


\begin{titlepage}
    \centering
    \includegraphics[width=0.65\textwidth]{logos/tu.pdf}\par\vspace{2cm}
    \vspace{3cm}
    {\scshape\fontsize{20pt}{21pt}\selectfont\textbf{Seminar: Tandem Project}\par}
    \vspace{0.5cm}
    {\scshape\fontsize{14pt}{16pt}\selectfont\textbf{My part of the report}\par}
    \vspace{10cm}
    {\scshape\Large Technische Universität Dortmund \par}
    \vspace{1cm}
    {\normalsize Nicole Schulte \par}
    {\normalsize nicole.schulte@tu-dortmund.de\par}
    \vspace{2cm}
  \end{titlepage}

\tableofcontents
\thispagestyle{empty}
\newpage
\section{Results}
The results are organized as follows. First, a study on the cross section as a function of the centre-of-mass energy is presented using two different 
\texttt{MadGraph} models. This study gives important validation to the results in 
the coefficient studies. Then, the analysis with one coefficient fixed to zero while the other varies is discussed. \texttt{MadGraph} results are displayed and then interpreted with 
the \texttt{EFTfitter}. The study is then repeated for both coefficients filling the whole parameter space. The report presents the findings explicitly for a centre-of-mass energy 
of $\SI{345}{\giga\electronvolt}$. Results for centre-of-mass energies from $\SIrange{346}{349}{\giga\electronvolt}$ are presented in the appendix. A combined result for all measurement 
is presented.

\subsection{Cross Section Study}
\label{subsec:cross}
The behavior of the cross section as a function of the centre-of-mass energy is analysed with the \texttt{MadGraph} models \texttt{sm}, representing the standard model implementation, and 
the \texttt{dim$6$top$\_$LO$\_$UFO}, representing beyond-the-standard-model physics implementations. Default parameters from \texttt{MadGraph} for the 
top-quark width $\text{w}_\text{t} = \SI{1.51}{\giga\electronvolt}$ and the top-quark mass $m_\text{t} = \SI{172.0}{\giga\electronvolt}$ are used. The 
results are displayed in Figure \ref{fig:cross_section}.

\begin{figure}[H]
\centering
\begin{subfigure}{.5\textwidth}
  \centering
  \includegraphics[width=\textwidth]{graphics/ctZ_0_CMS_cross_section_defaults.pdf}
\end{subfigure}%
\begin{subfigure}{.5\textwidth}
  \centering
  \includegraphics[width=\textwidth]{graphics/ctZ_1_CMS_cross_section_defaults.pdf}
\end{subfigure}
\caption{Cross section as a function of the centre-of-mass energy. The left diagram displays the behavior of the cross section in the \texttt{sm} $\left(\text{standard model} \right)$ 
implementation in \texttt{MadGraph} whereas the right diagram displays the \texttt{dim$6$top$\_$LO$\_$UFO} model implementation.}
\label{fig:cross_section}
\end{figure}

Compared to the results from the Conceptual Design Report, which are displayed in Figure \ref{fig:paperplot}, a different behavior of the cross section is found. The \texttt{MadGraph}
model \texttt{sm} does not show a resonant behaviour at a centre-of-mass energie double the top-quark mass as was expected. For higher centre-of-mass energies, 
the cross section rises significantely in contrast to the NNNLO calculations from the Conceptual Design Report \cite{CDR}. In the \texttt{dim$6$top$\_$LO$\_$UFO} model, centre-of-mass energies 
lower than two times the top-quark mass were not archievable due to the low phase space. Therefore, no resonance can be found. The increase in the 
cross section for higher centre-of-mass energies differs from the NNNLO expectations. 

\subsection{One Coefficient Analysis}
\label{subsec:one}
The study on one coefficient is performed by setting the other relevant coefficient $\left(\text{here namely}: C_\text{tZ} \,\text{or}\, C_\text{tw} \right)$ to zero. Figure \ref{fig:345_fit} shows 
the behavior of the cross section as a function of the coefficient strength. The simulated data from \texttt{MadGraph} is fitted with the polynomial functions \eqref{eqn:ctW_running} or 
\eqref{eqn:ctZ_running}. The fitting parameters for the different centre-of-mass energies are displayed in Table \ref{tab:Coefficients_Single}. The accuracy on the standard model value 
is taken from the Conceptual Design Report \cite{CDR} and scaled to the value of the simulation. The assumed accuracies are displayed in Table \ref{tab:uncertainties}.

\begin{align}
\sigma \left(C_\text{tW} \right) = \sigma_\text{SM} + \sigma_{\text{BSM},1} C_\text{tW} + \sigma_{\text{BSM},2} C_\text{tW}^2
\label{eqn:ctW_running}
\end{align}

\begin{align}
\sigma \left(C_\text{tZ} \right) = \sigma_\text{SM} + \sigma_{\text{BSM},1} C_\text{tZ} + \sigma_{\text{BSM},2} C_\text{tZ}^2
\label{eqn:ctZ_running}
\end{align}



\begin{figure}[H]
\centering
\begin{subfigure}{.5\textwidth}
  \centering
  \includegraphics[width=\textwidth]{graphics/345_ctW_cross_section.pdf}
\end{subfigure}%
\begin{subfigure}{.5\textwidth}
  \centering
  \includegraphics[width=\textwidth]{graphics/345_ctZ_cross_section.pdf}
\end{subfigure}
\caption{Cross section as a function of the coefficient strength $C_\text{tW}$ $\left(\text{left} \right)$ or $C_\text{tZ}$ $\left(\text{right} \right)$ for a centre-of-mass energy of 
$\SI{345}{\giga\electronvolt}$. The accuracy on the standard model 
value is taken from Table \ref{tab:uncertainties}. The fitting is performed with \texttt{curve$\_$fit}.}
\label{fig:345_fit}
\end{figure}

The functions \eqref{eqn:ctW_running} and \eqref{eqn:ctZ_running} with the corresponding parameters from \texttt{curve$\_$fit} are then given to the \texttt{EFTfitter} as input parameters. 
The standard model value returned from \texttt{MadGraph} along with the accuracy from the Conceptual Design Report from Table \ref{tab:uncertainties} are also implemented. A uniformal 
function is assumed as the prior, distributing equal amounts of probability to all coefficient strength possible in the area from $-10$ to $10$. Figure \ref{fig:EFT_345} displayes 
the output from the \texttt{EFTfitter}. 

\begin{figure}[H]
\centering
\begin{subfigure}{.5\textwidth}
  \centering
  \includegraphics[width=\textwidth]{graphics/345_ctW_only_final.pdf}
\end{subfigure}%
\begin{subfigure}{.5\textwidth}
  \centering
  \includegraphics[width=\textwidth]{graphics/345_ctZ_only_final.pdf}
\end{subfigure}
\caption{\texttt{EFTfitter} \cite{EFTFitter} results for $C_\text{tW}$ $\left(\text{left} \right)$ or $C_\text{tZ}$ $\left(\text{right} \right)$ for a centre-of-mass energy of 
$\SI{345}{\giga\electronvolt}$. The posterior function as a funtion of the coefficient strength is displayed on the vertical axis and the coefficient strength itself is 
displayed on the horizontal axis. The standard deviation of the marginalised posterior is displayed in a color scheme.}
\label{fig:EFT_345}
\end{figure}

The peaks of the posterior functions are located, where the standard model assumption 
crosses the fitting function in Figure \ref{fig:345_fit}. It can be seen, that the areas of the smallest intervals 
are very narrow, as assumed due to the high precision and accuracy of the FCC.





\subsection{Two Coefficient Analysis}
\label{subsec:two}
In this section, both $C_\text{tW}$ and $C_\text{tZ}$ are varied over every possible iteration of coefficient strengths from $-4$ to $4$ in 
$0.5$ steps. The data from \texttt{MadGraph} is fitted with equation \eqref{eqn:interference}. The values of the 
fitting parameters are displayed in Table \ref{tab:Coefficients_Interference}. 

\begin{align}
\sigma \left(C_\text{tZ}, C_\text{tW} \right) = \sigma_\text{SM} + \sigma_1 C_\text{tZ} + \sigma_2 C_\text{tW} + \sigma_3 C_\text{tW} C_\text{tZ} + \sigma_4 C_\text{tW}^2 + \sigma_5 C_\text{tZ}^2
\label{eqn:interference}
\end{align}

Figure \ref{fig:345_interference} displays the \texttt{EFTfitter} output for a centre-of-mass energy of $\SI{345}{\giga\electronvolt}$. The diagrams on the main diagonal display the same analysis as 
the figures from the one coefficient analysis, with the difference, that interference from the other coefficient is taken into account. Therefore, the posterior function of the coefficient strength 
also distributes in the area between the two main peaks. The diagrams on the secondary diagonal differ in the display option. The lower left diagram combines the results from 
the posterior function as a function of the coefficient strength from the main diagonal. The coefficient strength 
are displayed on the horizontal and vertical axis with the posterior function as the third axis or in this case the 
display option. In the phase space of the coefficient the probability of the coefficient stregth is marked as a color scheme according 
to the interval. The upper right diagram choses a heat map as the form of display option. Aras not colored can be excluded, with no probability 
distributed to them.

\begin{figure}[H]
\centering
  \includegraphics[width=\textwidth]{graphics/345_interference_final.pdf}
\caption{\texttt{EFTfitter} \cite{EFTFitter} results for the parameter space of both $C_\text{tW}$ and $C_\text{tZ}$ varying at a centre-of-mass energy of $\SI{345}{\giga\electronvolt}$. The upper left diagram displays the posterior function 
of $C_\text{tZ}$ as a function of $C_\text{tZ}$ with the standard deviation of the marginalised posterior. The lower right corner displays the same for $C_\text{tW}$. The upper right diagram 
shows a heat map of the posterior function in the phase space of the coeffient strengths. The lower right diagram displays the same but uses 
the smallest intervalls as a display option.}
\label{fig:345_interference}
\end{figure}



\subsubsection{Combined Results}
\label{subsubsec:combined}
The one and two coefficient analysis are performed for the centre-of-mass energies from $\SIrange{346}{349}{\giga\electronvolt}$. The parameters of the fitting functions from each measurement 
are given to the \texttt{EFTfitter} with their standard model values and accuracies. A combined analysis is performed and the output is displayed in Figure \ref{fig:combined}. It can be seen, 
that the shape of the ring loses intensity in the lower shoulder. The $\SI{68.61}{\percent}$ interval becomes smaller in 
the regions outside of the standard model value compared to the single measurement from Figure \ref{fig:345_interference}.
Figure \ref{fig:contour} displays the contribution each measurement has on the final ring shape displayed in the lower left diagram of Figure \ref{fig:combined}. The fully coloured area 
corresponds to the smallest $\SI{68.610}{\percent}$ interval that remains after each measurement narrows down the probability.

\begin{figure}[H]
\centering
  \includegraphics[width=\textwidth]{graphics/Combined_results_final.pdf}
\caption{\texttt{EFTfitter} \cite{EFTFitter} results for the combination of the measurements from $\SIrange{345}{349}{\giga\electronvolt}$. In comparison to 
Figure \ref{fig:345_interference} it can be seen, that the area of the smallest $\SI{68.610}{\percent}$ intervals narrows around the 
standard model value. The ring shape of the posterior function in the phase space of the parameters loses intensity 
in lower regions.}
\label{fig:combined}
\end{figure} 


\begin{figure}[H]
\centering
  \includegraphics[width=0.5\textwidth]{graphics/Contours_final.pdf}
\caption{Contour Plot performed with the \texttt{EFTfitter} \cite{EFTFitter}. The contribution of each measurement from 
various centre-of-mass energies from $\SIrange{345}{349}{\giga\electronvolt}$ are displayed in different colours. The smallest smallest $\SI{68.610}{\percent}$ interval
that cannot be excluded in either measurement is displayed as the filled area.}
\label{fig:contour}
\end{figure} 

\newpage
\section{Conclusion}

The cross section study in section \ref{subsec:cross} shows, that \texttt{MadGraph} 
does not provide a model yet, that displays the expected resonant behavior of the 
cross section as a function of the centre-of-mass energy. \texttt{MadGraph} is able 
to calculate at next-to-leading order. To archieve results comparable to the 
results from the Conceptual Design Report \cite{CDR}, a scaling to NNNLO needs 
to be performed. Due to the difference in the obtained and expected behavior of 
the cross section, further results in the coefficient studies need to be taken 
with caution. \par \medskip

The results from the coefficient studies display the incredible precision and 
accuracy that the Future Circular Collider is able to archieve. Using measurements and 
data recorded at the FCC, a sigificant amount of beyond-the-standard-model physics 
could be researched to higher accuracy that nowadays. More exclusions and 
more narrow ranges in the coefficient strength of 
$C_\text{tW}$ and $C_\text{tZ}$ can be made with multiple measurements included. 
Therefore it would be interesting to combine recent measurements with 
predictions from the Future Circular Collider. \par \medskip

The Future Circular Collider as a possible future accelerator provides many 
opportunities to test the standard model further. The coefficient study in this 
report only shows one of many possiblities that can be realized.
\newpage
\section{Appendix}

\textbf{A Information from the Conceptual Design Report}

\begin{figure}[H]
\centering
  \includegraphics[width=0.5\textwidth]{graphics/Plot_from_Paper.png}
\caption{Production cross section on the top-quark pair production from the Conceptual Design Report \cite{CDR}. Variations in the top-quark width and mass are displayed as bands around 
the default parameters. The accuracy on the cross section datapoints for different centre-of-mass energies are denoted as black errorbars. The calculation is performed at NNNLO. Results on 
the accuracy are scaled to this measurement and used this analysis.}
\label{fig:paperplot}
\end{figure}

\begin{table}
    \centering
    \caption{Accuracies taken from the FCC Conceptual Design Report \cite{CDR} at centre-of-mass energies from $\SIrange{345}{349}{\giga\electronvolt}$. The values are then applied on the 
    standard model value recieved from \texttt{MadGraph} \cite{MadGraph} for each measurement to recieve the accuracy band displayed i.e. in Figure \ref{fig:EFT_345}.}
    \label{tab:uncertainties}
    \begin{tabular}{S S S} 
        \toprule 
        {$\sqrt{s}$/$\left[ \SI{}{\giga\electronvolt} \right]$} & {Accuracy on $\sigma_\text{SM}$} \\
        \midrule
        \SI{345}{} & \SI{3.31}{\percent}  \\
        \SI{346}{} & \SI{2.87}{\percent}  \\
        \SI{347}{} & \SI{3.86}{\percent}  \\
        \SI{348}{} & \SI{3.77}{\percent}  \\
        \SI{349}{} & \SI{3.74}{\percent}  \\
        \bottomrule
    \end{tabular}
\end{table}



\textbf{B One Coefficient Analysis for $\sqrt{\text{s}} = \SIrange{346}{349}{\giga\electronvolt}$}
\begin{table}
    \centering
    \caption{Parameters of the fitting function \eqref{eqn:ctW_running} for various $C_\text{tW}$ with $C_\text{tZ}$ being constrained to zero $\left(\text{left} \right)$ as well as 
    the parameters of the fitting function \eqref{eqn:ctZ_running} for various $C_\text{tZ}$ and $C_\text{tW}$ being constrained to zero $\left(\text{right} \right)$. The fitting is performed 
    at centre-of-mass energies from $\SIrange{345}{349}{\giga\electronvolt}$. The fitting was performed with \texttt{curve$\_$fit} \cite{curve}.}
    \label{tab:Coefficients_Single}
    \resizebox{\textwidth}{!}{\begin{tabular}{S[table-format=3.1] S S S S S S} 
        \toprule 
        &  \multicolumn{3}{c}{$C_\text{tW}$ running, $C_\text{tZ} = 0$} & \multicolumn{3}{c}{$C_\text{tZ}$ running, $C_\text{tW} = 0$ } \\ 
        \cmidrule(lr){2-4}\cmidrule(lr){5-7}
        {$\sqrt{s}$/$\left[ \SI{}{\giga\electronvolt} \right]$} & {$\sigma_\text{SM}$/$\left[ \SI{}{\pico\barn} \right]$} & {$\sigma_{\text{BSM},1}$/$\left[ \SI{}{\pico\barn} \right]$} &  {$\sigma_{\text{BSM},2}$/$\left[ \SI{}{\pico\barn} \right]$} &  {$\sigma_{\text{SM}}$/$\left[ \SI{}{\pico\barn} \right]$} & {$\sigma_{\text{BSM},1}$/$\left[ \SI{}{\pico\barn} \right]$} & {$\sigma_{\text{BSM},2}$/$\left[ \SI{}{\pico\barn} \right]$} \\
        \midrule
        \SI{345}{}   & \SI{0.1368 \pm 0.0004}{} & \SI{0.3074 \pm 0.0007}{} & \SI{0.1802 \pm 0.0003}{} & \SI{0.1372 \pm 0.0002}{} & \SI{-0.2439 \pm 0.0005}{} & \SI{0.1489 \pm 0.0002}{} \\
        \SI{346}{}   & \SI{0.1977 \pm 0.0005}{} & \SI{0.4392 \pm 0.0010}{} & \SI{0.2570 \pm 0.0004}{} & \SI{0.1983 \pm 0.0003}{} & \SI{-0.3424 \pm 0.0005}{} & \SI{0.2086 \pm 0.0003}{} \\
        \SI{347}{}   & \SI{0.2399 \pm 0.0007}{} & \SI{0.5335 \pm 0.0145}{} & \SI{0.3128 \pm 0.0007}{} & \SI{0.2407 \pm 0.0003}{} & \SI{-0.4163 \pm 0.0007}{} & \SI{0.2541 \pm 0.0003}{} \\
        \SI{348}{}   & \SI{0.2754 \pm 0.0009}{} & \SI{0.6124 \pm 0.0017}{} & \SI{0.3593 \pm 0.0008}{} & \SI{0.2754 \pm 0.0004}{} & \SI{-0.4769 \pm 0.0008}{} & \SI{0.2917 \pm 0.0004}{} \\
        \SI{349}{}   & \SI{0.3051 \pm 0.0008}{} & \SI{0.6794 \pm 0.0017}{} & \SI{0.3996 \pm 0.0008}{} & \SI{0.2961 \pm 0.0005}{} & \SI{-0.5285 \pm 0.0011}{} & \SI{0.3249 \pm 0.0005}{} \\
        \bottomrule
    \end{tabular}}
\end{table}


\begin{figure}[H]
\centering
\begin{subfigure}{.5\textwidth}
  \centering
  \includegraphics[width=\textwidth]{graphics/346_ctW_cross_section.pdf}
\end{subfigure}%
\begin{subfigure}{.5\textwidth}
  \centering
  \includegraphics[width=\textwidth]{graphics/346_ctZ_cross_section.pdf}
\end{subfigure}
\caption{Cross section as a function of the coefficient strength $C_\text{tW}$ $\left(\text{left} \right)$ or $C_\text{tZ}$ $\left(\text{right} \right)$ for a centre-of-mass energy of 
$\SI{346}{\giga\electronvolt}$. The accuracy on the standard model 
value is taken from Table \ref{tab:uncertainties}. The fitting is performed with \texttt{curve$\_$fit}.}
\label{fig:346_cross}
\end{figure}


\begin{figure}[H]
\centering
\begin{subfigure}{.5\textwidth}
  \centering
  \includegraphics[width=\textwidth]{graphics/347_ctW_cross_section.pdf}
\end{subfigure}%
\begin{subfigure}{.5\textwidth}
  \centering
  \includegraphics[width=\textwidth]{graphics/347_ctZ_cross_section.pdf}
\end{subfigure}
\caption{Cross section as a function of the coefficient strength $C_\text{tW}$ $\left(\text{left} \right)$ or $C_\text{tZ}$ $\left(\text{right} \right)$ for a centre-of-mass energy of 
$\SI{347}{\giga\electronvolt}$. The accuracy on the standard model 
value is taken from Table \ref{tab:uncertainties}. The fitting is performed with \texttt{curve$\_$fit}.}
\label{fig:347_cross}
\end{figure}

\begin{figure}[H]
\centering
\begin{subfigure}{.5\textwidth}
  \centering
  \includegraphics[width=\textwidth]{graphics/348_ctW_cross_section.pdf}
\end{subfigure}%
\begin{subfigure}{.5\textwidth}
  \centering
  \includegraphics[width=\textwidth]{graphics/348_ctZ_cross_section.pdf}
\end{subfigure}
\caption{Cross section as a function of the coefficient strength $C_\text{tW}$ $\left(\text{left} \right)$ or $C_\text{tZ}$ $\left(\text{right} \right)$ for a centre-of-mass energy of 
$\SI{348}{\giga\electronvolt}$. The accuracy on the standard model 
value is taken from Table \ref{tab:uncertainties}. The fitting is performed with \texttt{curve$\_$fit}.}
\label{fig:348_cross}
\end{figure}

\begin{figure}[H]
\centering
\begin{subfigure}{.5\textwidth}
  \centering
  \includegraphics[width=\textwidth]{graphics/349_ctW_cross_section.pdf}
\end{subfigure}%
\begin{subfigure}{.5\textwidth}
  \centering
  \includegraphics[width=\textwidth]{graphics/349_ctZ_cross_section.pdf}
\end{subfigure}
\caption{Cross section as a function of the coefficient strength $C_\text{tW}$ $\left(\text{left} \right)$ or $C_\text{tZ}$ $\left(\text{right} \right)$ for a centre-of-mass energy of 
$\SI{349}{\giga\electronvolt}$. The accuracy on the standard model 
value is taken from Table \ref{tab:uncertainties}. The fitting is performed with \texttt{curve$\_$fit}.}
\label{fig:349_cross}
\end{figure}





\begin{figure}[H]
\centering
\begin{subfigure}{.5\textwidth}
  \centering
  \includegraphics[width=\textwidth]{graphics/346_ctW_only_final.pdf}
\end{subfigure}%
\begin{subfigure}{.5\textwidth}
  \centering
  \includegraphics[width=\textwidth]{graphics/346_ctZ_only_final.pdf}
\end{subfigure}
\caption{\texttt{EFTfitter} \cite{EFTFitter} results for $C_\text{tW}$ $\left(\text{left} \right)$ or $C_\text{tZ}$ $\left(\text{right} \right)$ for a centre-of-mass energy of 
$\SI{346}{\giga\electronvolt}$. The posterior function as a funtion of the coefficient strength is displayed on the vertical axis and the coefficient strength itself is 
displayed on the horizontal axis. The standard deviation of the marginalised posterior is displayed in a color scheme.}
\label{fig:346_eft}
\end{figure}


\begin{figure}[H]
\centering
\begin{subfigure}{.5\textwidth}
  \centering
  \includegraphics[width=\textwidth]{graphics/347_ctW_only_final.pdf}
\end{subfigure}%
\begin{subfigure}{.5\textwidth}
  \centering
  \includegraphics[width=\textwidth]{graphics/347_ctZ_only_final.pdf}
\end{subfigure}
\caption{\texttt{EFTfitter} \cite{EFTFitter} results for $C_\text{tW}$ $\left(\text{left} \right)$ or $C_\text{tZ}$ $\left(\text{right} \right)$ for a centre-of-mass energy of 
$\SI{347}{\giga\electronvolt}$. The posterior function as a funtion of the coefficient strength is displayed on the vertical axis and the coefficient strength itself is 
displayed on the horizontal axis. The standard deviation of the marginalised posterior is displayed in a color scheme.}
\label{fig:347_eft}
\end{figure}

\begin{figure}[H]
\centering
\begin{subfigure}{.5\textwidth}
  \centering
  \includegraphics[width=\textwidth]{graphics/348_ctW_only_final.pdf}
\end{subfigure}%
\begin{subfigure}{.5\textwidth}
  \centering
  \includegraphics[width=\textwidth]{graphics/348_ctZ_only_final.pdf}
\end{subfigure}
\caption{\texttt{EFTfitter} \cite{EFTFitter} results for $C_\text{tW}$ $\left(\text{left} \right)$ or $C_\text{tZ}$ $\left(\text{right} \right)$ for a centre-of-mass energy of 
$\SI{348}{\giga\electronvolt}$. The posterior function as a funtion of the coefficient strength is displayed on the vertical axis and the coefficient strength itself is 
displayed on the horizontal axis. The standard deviation of the marginalised posterior is displayed in a color scheme.}
\label{fig:348_eft}
\end{figure}

\begin{figure}[H]
\centering
\begin{subfigure}{.5\textwidth}
  \centering
  \includegraphics[width=\textwidth]{graphics/349_ctW_only_final.pdf}
\end{subfigure}%
\begin{subfigure}{.5\textwidth}
  \centering
  \includegraphics[width=\textwidth]{graphics/349_ctZ_only_final.pdf}
\end{subfigure}
\caption{\texttt{EFTfitter} \cite{EFTFitter} results for $C_\text{tW}$ $\left(\text{left} \right)$ or $C_\text{tZ}$ $\left(\text{right} \right)$ for a centre-of-mass energy of 
$\SI{349}{\giga\electronvolt}$. The posterior function as a funtion of the coefficient strength is displayed on the vertical axis and the coefficient strength itself is 
displayed on the horizontal axis. The standard deviation of the marginalised posterior is displayed in a color scheme.}
\label{fig:349_eft}
\end{figure}






\textbf{C Two Coefficient Analysis for $\sqrt{\text{s}} = \SIrange{346}{349}{\giga\electronvolt}$}

\begin{table}
    \centering
    \caption{Parameters of the fitting function \eqref{eqn:interference} for $C_\text{tW}$ and $C_\text{tZ}$ running, filling up the whole parameter space. The fitting is performed 
    at centre-of-mass energies from $\SIrange{345}{349}{\giga\electronvolt}$. The fitting was performed with \texttt{curve$\_$fit} \cite{curve}.}
    \label{tab:Coefficients_Interference}
    \resizebox{\textwidth}{!}{\begin{tabular}{S S S S S S S} 
        \toprule 
        {$\sqrt{s}$/$\left[ \SI{}{\giga\electronvolt} \right]$} & {$\sigma_\text{SM}$/$\left[ \SI{}{\pico\barn} \right]$} & {$\sigma_1$/$\left[ \SI{}{\pico\barn} \right]$} & {$\sigma_2$/$\left[ \SI{}{\pico\barn} \right]$} &  {$\sigma_{3}$/$\left[ \SI{}{\pico\barn} \right]$} &  {$\sigma_{4}$/$\left[ \SI{}{\pico\barn} \right]$} & {$\sigma_{5}$/$\left[ \SI{}{\pico\barn} \right]$} \\
        \midrule
        \SI{345}{}   & \SI{0.1382 \pm 0.0020}{} & \SI{0.3065 \pm 0.0005}{} & \SI{-0.2424 \pm 0.0005}{} & \SI{-0.3064 \pm 0.0002}{} & \SI{0.1477 \pm 0.0002}{} & \SI{0.1793 \pm 0.0002}{} \\
        \SI{346}{}   & \SI{0.1993 \pm 0.0027}{} & \SI{0.4382 \pm 0.0007}{} & \SI{-0.3412 \pm 0.0007}{} & \SI{-0.4337 \pm 0.0003}{} & \SI{0.2075 \pm 0.0003}{} & \SI{0.2557 \pm 0.0003}{} \\
        \SI{347}{}   & \SI{0.2417 \pm 0.0034}{} & \SI{0.5325 \pm 0.0008}{} & \SI{-0.4144 \pm 0.0008}{} & \SI{-0.5279 \pm 0.0003}{} & \SI{0.2527 \pm 0.0004}{} & \SI{0.3113 \pm 0.0004}{} \\
        \SI{348}{}   & \SI{0.2755 \pm 0.0040}{} & \SI{0.6107 \pm 0.0010}{} & \SI{-0.4748 \pm 0.0004}{} & \SI{-0.6061 \pm 0.0004}{} & \SI{0.2904 \pm 0.0004}{} & \SI{0.3572 \pm 0.0004}{} \\
        \SI{349}{}   & \SI{0.2980 \pm 0.0050}{} & \SI{0.6643 \pm 0.0012}{} & \SI{-0.5261 \pm 0.0012}{} & \SI{-0.6691 \pm 0.0005}{} & \SI{0.3229 \pm 0.0006}{} & \SI{0.3911 \pm 0.0006}{} \\
        \bottomrule
    \end{tabular}}
\end{table}


\begin{figure}[H]
\centering
  \includegraphics[width=\textwidth]{graphics/346_interference_final.pdf}
\caption{\texttt{EFTfitter} \cite{EFTFitter} results for the parameter space of both $C_\text{tW}$ and $C_\text{tZ}$ varying at a centre-of-mass energy of $\SI{346}{\giga\electronvolt}$. The upper left diagram displays the posterior function 
of $C_\text{tZ}$ as a function of $C_\text{tZ}$ with the standard deviation of the marginalised posterior. The lower right corner displays the same for $C_\text{tW}$. The upper right diagram 
shows a heat map of the posterior function in the phase space of the coeffient strengths. The lower right diagram displays the same but uses 
the smallest intervalls as a display option.}
\label{fig:346_int_eft}
\end{figure}

\begin{figure}[H]
\centering
  \includegraphics[width=\textwidth]{graphics/347_interference_final.pdf}
\caption{\texttt{EFTfitter} \cite{EFTFitter} results for the parameter space of both $C_\text{tW}$ and $C_\text{tZ}$ varying at a centre-of-mass energy of $\SI{347}{\giga\electronvolt}$. The upper left diagram displays the posterior function 
of $C_\text{tZ}$ as a function of $C_\text{tZ}$ with the standard deviation of the marginalised posterior. The lower right corner displays the same for $C_\text{tW}$. The upper right diagram 
shows a heat map of the posterior function in the phase space of the coeffient strengths. The lower right diagram displays the same but uses 
the smallest intervalls as a display option.}
\label{fig:347_int_eft}
\end{figure}

\begin{figure}[H]
\centering
  \includegraphics[width=\textwidth]{graphics/348_interference_final.pdf}
\caption{\texttt{EFTfitter} \cite{EFTFitter} results for the parameter space of both $C_\text{tW}$ and $C_\text{tZ}$ varying at a centre-of-mass energy of $\SI{348}{\giga\electronvolt}$. The upper left diagram displays the posterior function 
of $C_\text{tZ}$ as a function of $C_\text{tZ}$ with the standard deviation of the marginalised posterior. The lower right corner displays the same for $C_\text{tW}$. The upper right diagram 
shows a heat map of the posterior function in the phase space of the coeffient strengths. The lower right diagram displays the same but uses 
the smallest intervalls as a display option.}
\label{fig:348_int_eft}
\end{figure}

\begin{figure}[H]
\centering
  \includegraphics[width=\textwidth]{graphics/349_interference_final.pdf}
\caption{\texttt{EFTfitter} \cite{EFTFitter} results for the parameter space of both $C_\text{tW}$ and $C_\text{tZ}$ varying at a centre-of-mass energy of $\SI{349}{\giga\electronvolt}$. The upper left diagram displays the posterior function 
of $C_\text{tZ}$ as a function of $C_\text{tZ}$ with the standard deviation of the marginalised posterior. The lower right corner displays the same for $C_\text{tW}$. The upper right diagram 
shows a heat map of the posterior function in the phase space of the coeffient strengths. The lower right diagram displays the same but uses 
the smallest intervalls as a display option.}
\label{fig:349_int_eft}
\end{figure}
\newpage
\printbibliography
\end{document}